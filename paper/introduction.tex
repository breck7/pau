\section{Introduction} \label{sec:intro}

\iffalse
Outline and Notes

Medical records went electronic for a reason
	- portability
	- resilience
	- accessibility
	- etc...
accessible for who?
	- only appears to help care providers and admin within a hospital within
	  a specific EMR domain
	- if we imagine an information chain, patient access is limited
	- researcher access is limited
Biomedical field is in the big data era thanks to EMR
	- big data affords many tools such as machine and deep learning 
	- EMR inconsistencies is a problem
		- this is why data curration exists
		- different systems
		- talk about the contrast of using HL7 in the back end of most
		  systems but output is different for each system
Acquiring clean and good quality medical data is important w.r.t medical and
healthcare advancement

\fi

Medical records began going electronic in the mid 1960s~\cite{TODO} and for
good reason.  Electronic medical records (EMR) are often viewed as superior to
paper or hard copy records because EMR's are more portable, more resilient and
more accessible.  In addition, modern hospital infrastructure is set up such to support
EMR systems and provide easy of use for healthcare providers.  While EMR's
resulted in massive improvements to healthcare, issues with respect to access
to EMR data appear to be a bottleneck for further healthcare benefits.  
