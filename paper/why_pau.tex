\section{Why Pau} \label{sec:why_pau}

%\subsection{Non disruptive to current hospital workflows}
%\subsection{Easy curration}
%\subsection{File format agnostic}
\iffalse
Advantage of Pau:
        - hospitals and healthcare systems can continue their current workflow
                - charting and EMR entry can continue as usual
                - clinicians do not have to learn a new tool
    - Tree Notation Works with paper offline as well.
        - grammars, defined for each EMR system, can convert everything into a
          consistent tree notation format
                - efficiently aggregate/curate data
                - medical field has entered the big data era
                        - data consistency is important for big data analysis
                                - e.g. machine/deep learning model construction
    - Grammars can be Concatenated by simply concatenating 2 grammar files
     - systems can use only the grammars for the target domain
     - multiomics systems can use all grammars at once.
        * include code snipits and examples of Pau
                - or include example of tree
                        - the toCSV() example in the sandbox is very practice
                          and applicable to EMR
  - Omnifix notation
  - Grammars can be used to generate mock data
   - Code can be written and tested against mock data, and then run on data held
     in a black box to safeguard privacy.
  - Internationalization
   - i18n'izing each grammar is as simple as adding 1 word per concept
  - Wikipedia like crowdsourcing of a global grammar for EMR
\fi

