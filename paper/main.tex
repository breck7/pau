%
% File main.tex
%
% Contact: , 
%%
%%
%%  
%% 

%\documentclass{IEEEtran}
\documentclass[journal]{IEEEtran}
%\documentclass[10pt]{article}
\newcommand{\sansserifformat}[1]{\fontfamily{cmss}{ #1}}%

%\usepackage[letterpaper]{geometry}
\usepackage{times}
\usepackage[none]{hyphenat}
\usepackage{url}
\usepackage{latexsym}
%\usepackage{minted}
%\usepackage{indentfirst}
\usepackage{graphicx}
\graphicspath{{figures/}}
\usepackage{amsmath}
\usepackage{amssymb}
\usepackage{enumitem}
\usepackage{setspace}
\usepackage{paralist}
\usepackage{bm}
\usepackage{cite}
\usepackage{multirow}
\usepackage{todonotes}
\usepackage{umoline}
\usepackage{xspace}
\usepackage{soul}
\usepackage{graphicx}
\usepackage{url}
\usepackage{subfigure}
%\usepackage{subcaption}

%\setlength\titlebox{5cm}

\begin{document}
\title{Pau, A tree language for electronic medical records}

% TODO: change for final
\iffalse
\author{
  Lambert Leong \\
  University of Hawaii \\
  {\underline{lambert3@hawaii.edu}} \\
\and
  Breck Yunits\\
  University of Hawaii Cancer Center\\
  {\underline{breckuh@gmail.com}}\\ }
\fi


\date{}

\maketitle
\begin{abstract}
	TODO: abstract
\end{abstract}

\iffalse
BASIC OUTLINE and NOTES

Define the problem: There is no consistent format or template for medical records
	- often electronic medical records (EMR) are stored in various forms and
	  formats
		- e.g. csv, xls, sas, etc...
	- there was a big push to have all hospitals use the same systems
		- has not been done and does not look like it will happen soon
		- resistance from users, administrators, stakeholders

Offer solution: Introduce Pau and Tree Notation
	- define a grammar for your tree language with respect to each EMR system
	- NLP tools, possibly powered by machine/deep learning, can help to
	  quickly define grammars and languages for EMR systems

Advantage of Pau:
	- hospitals and healthcare systems can continue their current workflow
		- charting and EMR entry can continue as usual 
		- clinicians do not have to learn a new tool
	- grammars, defined for each EMR system, can convert everything into a
	  consistant tree notation format
		- efficiently aggrigate/curate data 
		- medical field has entered the big data era
			- data consistency is important for big data analysis
				- e.g. machine/deep learning model construction
	* include code snipits and examples of Pau
		- or include example of tree
			- the toCSV() example in the sandbox is very practicle
			  and applicable to EMR

On going and future work
	- grammars for particular EMR systems are being defined
		- tools and techniques to optimize this process being worked on
		- a person/expert is needed to QA/QC the grammars etc
	- potential use cases and example of how it optimizes analysis pipelines
	- how Pau fits in the grand scheme or vision of tree notation
\fi

\section{Introduction} \label{sec:intro}
Place holder text.  Refer to outline in comment block in main.tex.

Example of citing tree paper~\cite{yunits2017tree}.

%\input{relatedwork}
%\input{methods}
%\input{results}
%\input{discussion}
%\section{Conclusion} \label{sec:conclusion}



%Bibliography 
\bibliographystyle{ieeetr}
\bibliography{bib}
%\addbibresource{bib.bib}


\end{document}
